%===========================================================
% File: chapters/chapter1.tex
% Title: Template User Guide (Final Version with English Comments)
%===========================================================

\chapter{راهنمای نگارش و استفاده از قالب}% کلیات پژوهش

\section{مقدمه}
این فصل به عنوان راهنمای عملی برای استفاده از این قالب نگارش شده است. هدف از ارائه این قالب، استانداردسازی پایان‌نامه‌ها و کاهش دغدغه‌های دانشجویان در صفحه‌آرایی است تا بتوانند بر محتوای علمی خود تمرکز کنند. در ادامه، نحوه استفاده از امکانات مختلف این قالب با ذکر مثال توضیح داده می‌شود.

\section{نحوه نگارش متن (فارسی و انگلیسی)}
این قالب بر پایه بسته قدرتمند \lr{XePersian} طراحی شده است. اصول کلی نگارش به شرح زیر است:

\begin{itemize}
	\item \textbf{متن فارسی:} به صورت عادی تایپ می‌شود و راست‌چین است.
	\item \textbf{کلمات انگلیسی وسط متن:} باید حتماً درون دستور \verb|\lr{...}| قرار گیرند.
	\item \textbf{پاراگراف کامل انگلیسی:} (مانند چکیده انگلیسی یا کدهای طولانی) باید درون محیط \texttt{latin} قرار گیرند.
\end{itemize}

\begin{ex}{مثال نگارش صحیح}{text_example}
	درست: برای پردازش تصویر از کتابخانه \lr{OpenCV} استفاده کردیم.\\
	نادرست: برای پردازش تصویر از کتابخانه \lr{OpenCV} استفاده کردیم. (این حالت باعث به‌هم‌ریختگی متن می‌شود).
\end{ex}

\section{فرمول‌نویسی ریاضی}
یکی از نقاط قوت \lr{\LaTeX}، کیفیت بالای فرمول‌های ریاضی است. در این قالب، شماره‌گذاری فرمول‌ها به صورت خودکار و فارسی انجام می‌شود.

\subsection{فرمول‌های درون‌خطی}
برای نوشتن فرمول در میان متن از علامت دلار ($\$$) استفاده کنید. 
مثال: رابطه فیثاغورس به صورت
 $a^2 + b^2 = c^2$
  است که در هندسه کاربرد دارد.

\subsection{فرمول‌های جداگانه (شماره‌دار)}
برای فرمول‌های مهم که باید در سطر جداگانه و با شماره باشند، از محیط \texttt{equation} استفاده کنید. برای ارجاع به فرمول، حتماً از \verb|\label{...}| استفاده کنید:

\begin{equation}
	\label{eq:fourier}
	f(x) = \int_{-\infty}^{\infty} \hat{f}(\xi)\,e^{2\pi i \xi x} \,d\xi
\end{equation}

همانطور که در رابطه \eqref{eq:fourier} مشاهده می‌کنید، ارجاع به فرمول‌ها نیز با دستور \verb|\eqref{...}| به راحتی انجام می‌شود و شماره فرمول به صورت خودکار لینک می‌شود.

\section{جداول و تصاویر}

\subsection{درج تصاویر}
برای قرار دادن تصویر، از محیط \texttt{figure} استفاده کنید. تصاویر باید در پوشه \lr{\texttt{images}} ذخیره شده باشند. همیشه برای تصاویر \verb|\caption| و \verb|\label| تعریف کنید.

\begin{figure}[ht]
	\centering
	% Replace 'logo.png' with your own image file name
	\includegraphics[width=0.3\textwidth]{./images/logo.png}
	\caption{نمونه درج تصویر لوگوی دانشگاه}
	\label{fig:logo_sample}
\end{figure}

با توجه به شکل \ref{fig:logo_sample}، توضیحات تصویر باید در پایین آن قرار گیرد.

\subsection{رسم جداول}
این قالب از بسته \lr{booktabs} برای خطوط زیبا و بسته \lr{xcolor} برای رنگ‌آمیزی سطرها پشتیبانی می‌کند.

\begin{table}[ht]
	\centering
	\caption{نمونه یک جدول استاندارد با سطرهای رنگی}
	\label{tab:sample_table}
	\small
	% Set row colors (alternating light gray and white)
	\rowcolors{2}{rowgray}{white}
	\begin{tabular}{c r c}
		\toprule
		\textbf{ردیف} & \textbf{متغیر} & \textbf{مقدار} \\
		\midrule
		۱ & فشار اولیه & $100$ \lr{Pa} \\
		۲ & دمای محیط & $25$ \lr{C} \\
		۳ & چگالی & $1.2$ \lr{g/cm} \\
		\bottomrule
	\end{tabular}
\end{table}

\section{باکس‌های اختصاصی قالب}
در این قالب، محیط‌های رنگی جذابی برای تعاریف، اهداف و نکات مهم تعریف شده است. ساختار کلی استفاده به صورت زیر است:\\
\verb|\begin{BoxName}{Title}{Label} ... \end{BoxName}|

\subsection{باکس تعریف (\lr{Definition})}
\begin{defn}{الگوریتم ازدحام ذرات}{def_pso}
	الگوریتم ازدحام ذرات یا \lr{PSO}، یک روش بهینه‌سازی فراابتکاری است که از رفتار اجتماعی دسته‌های پرندگان برای یافتن بهینه سراسری الهام گرفته شده است.
\end{defn}

\subsection{باکس اهداف (\lr{Purpose})}
\begin{purp}{اهداف پژوهش}{purp_thesis}
	\begin{enumerate}
		\item بررسی ادبیات موضوع و روش‌های پیشین.
		\item ارائه مدل ریاضی جدید برای زنجیره تأمین.
		\item حل مدل با استفاده از الگوریتم‌های فراابتکاری.
	\end{enumerate}
\end{purp}

\section{نحوه درج کد (\lr{Listing})}
برای نمایش کدهای برنامه‌نویسی، از محیط \texttt{lstlisting} استفاده کنید. این محیط در این قالب بهینه‌سازی شده تا کدهای طولانی را بشکند و از صفحه خارج نشود.

% Local settings to ensure correct display (LTR direction mandatory)
\begin{latin}
	\begin{lstlisting}[
		language=Python, 
		caption={\lr{Sample code-space}}, 
		label={code:python},
		basicstyle=\ttfamily\footnotesize, % Code font size
		breaklines=true, % Break long lines automatically
		frame=lines, % Frame around code
		numbers=left, % Line numbers on the left
		numberstyle=\tiny\color{gray},
		keywordstyle=\color{blue}\bfseries,
		stringstyle=\color{green!50!black}
		]
		def optimize(data):
		"""
		This function optimizes the input data using a simple
		multiplication heuristic. It handles large datasets efficiently.
		"""
		result = []
		for item in data:
		# A simple calculation demo
		calc = item * 2 + 100 
		result.append(calc)
		
		return result
	\end{lstlisting}
\end{latin}

همانطور که در قطعه‌کد \ref{code:python} مشاهده می‌کنید، توضیحات فارسی در کپشن کد امکان‌پذیر است، اما بدنه کد باید انگلیسی باشد.

\section{مدیریت مراجع}
رفرنس‌دهی در این قالب با استفاده از \lr{BibTeX} انجام می‌شود. فایل مراجع در \lr{\texttt{ref.bib}} قرار دارد.

\begin{itemize}
	\item \textbf{ارجاع داخل پرانتز:} دستور \verb|\citep{ref1}| $\leftarrow$ خروجی: (نام‌نویسنده، سال)
	\item \textbf{ارجاع در متن:} دستور \verb|\citet{ref1}| $\leftarrow$ خروجی: نام‌نویسنده (سال)
\end{itemize}


به طور مثال: 
 \citep{schumpeter1934}
 

فقط دقت کنید در فایل \lr{ref.bib}  یک لاین جدید اضافه کنید به نام: \lr{authorfa}  و داخلش همانند نمونه نام فارسی نویسنده را بنویسید. اگر چندین نویسنده داشتین بعد از نام هر نویسنده یک فاصله و واژه \lr{and} را بنویسید. نکته مهم اینکه در نوشتن نامهای فارسی فاصله بین کاراکتر‌های یک نام نباشد و به صورت نیم فاصله وارد کنید. مثلا: علی‌رضا.


استایل پیش‌فرض \lr{asa-fa} است، اما در فایل \lr{\texttt{main.tex}} راهنمای تغییر آن به سایر استایل‌ها (مانند \lr{IEEE} یا \lr{Chicago})
 موجود است.
 